\chapter{Reglas de transcripción}

\label{AppendixB}

En este apéndice se listan todas las reglas de transcripción que TexToES usa para poder obtener una transcripción final.

Del lado izquierdo esta la etiqueta CMathML que representa la entidad matemática mientras que del lado derecho está la transcripción.

\begin{lstlisting}[basicstyle=\scriptsize]
[constantes]
  infinity = infinito
  imaginaryi = I
  integers = conjunto de los enteros
  emptyset = conjunto vacio
  real = el conjunto de los reales

[operadores]
#Constructores
  list = la lista de $VAR$
  vector = el vector compuesto por $VAR$
  set = el conjunto de todos $VAR$ que cumplan que
    $bvar$ sea $condition$, el conjunto de $VAR$
  imaginary = Operacion imaginaria $VAR$

#Relaciones
  eq = $VAR$ es igual a $VAR$
  neq = $VAR$ no es igual a $VAR$
  gt = $VAR$ es mayor a $VAR$
  geq = $VAR$ es mayor o igual a $VAR$
  lt = $VAR$ es menor a $VAR$
  leq = $VAR$ es menor o igual a $VAR$
  equivalent = $VAR$ es equivalente a $VAR$
  approx = $VAR$ es aproximado a $VAR$
  factorof = $VAR$ divide a $VAR$

#Operadores basicos
  plus = $VAR$ mas $VAR$
  minus = $VAR$ menos $VAR$, menos $VAR$
  times = $VAR$ por $VAR$
  divide = $VAR$ dividido $VAR$
  power = $VAR$ elevado a la $VAR$
  exp = e elevado a la $VAR$
  root = raiz $degree$ de $VAR$

#Calculo
  sum = la sumatoria de $bvar$ que va desde
    $lowlimit$ hasta $uplimit$ de $VAR$
  product = la productoria de $bvar$ que
    va desde $lowlimit$ hasta $uplimit$ de $VAR$
  limit = el limite de $bvar$ que tiende
    $lowlimit$ de $VAR$

#Operadores logicos
  not = $VAR$ negado
  and = $VAR$ y $VAR$
  or = $VAR$ o $VAR$
  implies = $VAR$ implica $VAR$

#Conjuntos
  in = $VAR$ en $VAR$
  notin = $VAR$ no esta en $VAR$
  intersect = $VAR$ intersectado con el conjunto $VAR$
  union = $VAR$ union $VAR$
  setdiff = $VAR$ menos el conjunto $VAR$
  prsubset = $VAR$ es subconjunto de $VAR$

#Funciones conocidas
  function = $VAR$ de $VAR$
  inverse = $VAR$ inversa
  max = maximo entre $VAR$
  determinant = determinante de $VAR$
  gcd = maximo comun divisor entre $VAR$ y $VAR$
  lcm = minimo comun multiplo entre $VAR$ y $VAR$
  factorial = $VAR$ factorial
  ln = logaritmo natural de $VAR$
  log = logaritmo $base$ de $VAR$
  sin = seno de $VAR$
  arcsin = arcoseno de $VAR$
  cos = coseno de $VAR$
  arccos = arcocoseno de $VAR$
  tan = tangente de $VAR$
  arctan = arcotangente de $VAR$
  sinh = seno hiperbolico de $VAR$
  arcsinh = arcoseno hiperbolico de $VAR$
  cosh = coseno hiperbolico de $VAR$
  arccosh = arcocoseno hiperbolico de $VAR$
  tanh = tangente hiperbolica de $VAR$
  arctanh = arcotangente hiperbolica de $VAR$
  sech = secante hiperbolica de $VAR$
  arcsech = arcosecante hiperbolica de $VAR$
  csch = cosecante hiperbolica de $VAR$
  arccsch = arcocosecante hiperbolica de $VAR$
  coth = cotangente hiperbolica de $VAR$
  arccoth = arcocotangente hiperbolica de $VAR$
  cot = cotangente de $VAR$
\end{lstlisting}
