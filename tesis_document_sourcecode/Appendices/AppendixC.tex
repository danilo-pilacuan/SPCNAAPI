\chapter{Resultados de retrotraducción}

\label{AppendixC}

Aquí se listan las transcripciones usadas en el proceso de evaluación mediante la retrotraducción.
Del lado derecho de la tabla usted puede ver el latex utilizado para generar las transcripciones que estan en el lado izquierdo.

\begin{center}
 \begin{tabular}{||c  c||}
    \hline
    Transcripción & LaTeX usado \\ [0.5ex]
    \hline\hline
    (raiz 6 de 5) & $\sqrt[6]{5}$ \\ 
    \hline
    (X elevado a la (1 dividido 2))& $x^\frac{1}{2}$ \\
    \hline
    (A no esta en conjunto vacio)& $a \not\in \emptyset$ \\
    \hline
    (A union B)& $A \cup B$ \\
    \hline
    ((1 es menor o igual a X) y (X es menor a Y))& $1\leq x < y$ \\
    \hline
    (conjunto de los enteros no es igual a conjunto vacio)& $\mathbb{Z} \neq \emptyset$ \\
    \hline
    ((seno de X) mas (logaritmo natural de Y) mas (cotangente de Z))& $\sin x + \ln y + \cot z$ \\
    \hline
    (((3 multiplicado por X) menos 7) es igual a 0)& $3x - 7 = 0$ \\
    \hline
    (A multiplicado por B)& $A \times B$ \\
    \hline
    (A no es igual a B)& $A \neq B$ \\
    \hline
    (A intersectado con el conjunto B)& $A \cap B$ \\
    \hline
    (((P negado) negado) es igual a P)& $\lnot \lnot p = p$ \\
    \hline
    (1 dividido (3 mas 7))& $\frac{1}{3+7}$ \\
    \hline
    ((F aplicado a X) es igual a (X elevado a la 3)) & $f(x) = x^3$ \\[1ex]
 \hline
\end{tabular}
\end{center}

Recordemos que el objetivo de esta evaluación era pedirles a los voluntarios que escribieran la fórmula que les sugería la transcripción para luego compararlas con las fórmulas reales usadas.

Las respuestas en las siguientes tablas, $t_i$ significa la transcripción i, $v_i$ significa el voluntario i.

\begin{center}
\begin{tabular}{ |c|c|c|c|c|c|c|c| }
 \hline
    $v$ & $t_1$ & $t_2$ & $t_3$ & $t_4$ & $t_5$ & $t_6$ & $t_7$ \\[0.5ex]
    \hline\hline
    $v_1$ & $ \sqrt[6]{5}$ & $x^\frac{1}{2}$ & $a \not\in \emptyset$          & $A \cup B$ & $1\leq x \wedge x < y$    & $\mathbb{Z} \neq \emptyset$            & $\sin x + \ln y + \cot z$  \\
    $v_2$ &$ \sqrt[6]{5}$ & $x^\frac{1}{2}$ & $a \not\in \emptyset$          & $A \cup B$ & $1\leq x < y$             & $\{ \mathbb{Z} \} \neq \emptyset$      & $\sin x + \ln y + \cot z$  \\
    $v_3$ &$ \sqrt[6]{5}$ & $x^\frac{1}{2}$ & $a \not\in \emptyset$          & $A \cup B$ & $1\leq x \wedge x < y$    & $\{ \mathbb{Z} \} \neq \{\emptyset\}$  & $\sin x + \ln y + \cot z$  \\
    $v_4$ &$ \sqrt[6]{5}$ & $ \sqrt{2}$     & $a \not\in \emptyset$          & $A \cup B$ & $1\leq x \wedge x \leq y$ & $\mathbb{Z} \neq \emptyset$            & $\sin x + \ln y + \cot z$  \\
    $v_5$ &$ \sqrt[6]{5}$ & $x^{1 \div 2}$  & $a \not\in \left \{ \right \}$ & $A \cup B$ & $1\leq x \wedge x < y$    & $\mathbb{Z} \neq \left \{ \right \}$   & $\sin x + \ln y + \cot z$  \\
    $v_6$ &$ \sqrt[6]{5}$ & $x^\frac{1}{2}$ & $a \not\in \emptyset$          & $A \cup B$ & $1\leq x \wedge x < y$    & $\mathbb{Z} \neq \emptyset$            & $\sin x + \ln y + \cot z$  \\
    $v_7$ &$ \sqrt[6]{5}$ & $x^\frac{1}{2}$ & $a \not\in \emptyset$          & $A \cup B$ & $1\leq x \wedge x < y$    & $\mathbb{Z} \neq \emptyset$            & $\sin x + \ln y + \cot z$  \\
    $v_8$ &$ \sqrt[6]{5}$ & $x^\frac{1}{2}$ & $a \not\in \emptyset$          & $A \cup B$ & $1\leq x \wedge x < y$    & $\mathbb{Z} \neq \emptyset$            & $\sin x + \ln y + \cot z$  \\
    $v_9$ &$ \sqrt[6]{5}$ & $x^\frac{1}{2}$ & $a \not\in \emptyset$          & $A \cup B$ & $1\leq x \wedge x < y$    & $\mathbb{Z} \neq \emptyset$            & $\sin x + \ln y + \cot z$  \\
    $v_{10}$ &$ \sqrt[6]{5}$ & $x^\frac{1}{2}$ & $a \not\in \emptyset$          & $A \cup B$ & $1\leq x \wedge x > y$    & $\mathbb{Z} \neq \emptyset$            & $\sin x + \ln y + \cot z$  \\
    $v_{11}$ &$ \sqrt[6]{5}$ & $x^\frac{1}{2}$ & $a \not\in \emptyset$          & $A \cup B$ & $1\leq x \wedge x > y$    & $\mathbb{Z} \neq \emptyset$            & $\sin x + \ln y + \cot z$  \\
    $v_{12}$ &$ \sqrt[6]{5}$ & $x^\frac{1}{2}$ & $a \not\in \emptyset$          & $A \cup B$ & $1\leq x \wedge x > y$    & $\mathbb{Z} \neq \emptyset$            & $\sin x + \ln y + \cot z$  \\
    $v_{13}$ &$ \sqrt[6]{5}$ & $x^\frac{1}{2}$ & $a \not\in \{\emptyset\}$      & $A \cup B$ & $1\leq x \wedge x < y$    & $\mathbb{Z} \neq \emptyset$            & $\sin x + \ln y + \cot z$  \\
 \hline
\end{tabular}
\end{center}


\begin{center}
\begin{tabular}{ |c|c|c|c|c|c|c|c| }
 \hline
 $v$ & $t_8$        & $t_9$        & $t_10$     & $t_11$     & $t_12$              & $t_13$          & $t_14$ \\[0.5ex]
    \hline\hline
 $v_1$ & $3x - 7 = 0$ & $A \times B$ & $A \neq B$ & $A \cap B$ & $\lnot \lnot p = p$ & $\frac{1}{3+7}$ & $f(x) = x^3$ \\
 $v_2$ & $3x - 7 = 0$ & $A \times B$ & $A \neq B$ & $A \cap B$ & $\lnot \lnot p = p$ & $\frac{1}{3+7}$ & $f(x) = x^3$ \\
 $v_3$ & $3x - 7 = 0$ & $A \times B$ & $A \neq B$ & $A \cap B$ & $\lnot \lnot p = p$ & $\frac{1}{3+7}$ & $f(x) = x^3$ \\
 $v_4$ & $3x - 7 = 0$ & $A \times B$ & $A \neq B$ & $A \cap B$ & $\lnot \lnot p = p$ & $1\div(3+7)$    & $f(x) = x^3$ \\
 $v_5$ & $$           & $A \times B$ & $A \neq B$ & $A \cap B$ & $\lnot \lnot p = p$ & $1 \div (3+7)$  & $f(x) = x^3$ \\
 $v_6$ & $3x - 7 = 0$ & $A \times B$ & $A \neq B$ & $A \cap B$ & $\lnot \lnot p = p$ & $\frac{1}{3+7}$ & $f(x) = x^3$ \\
 $v_7$ & $3x - 7 = 0$ & $A \times B$ & $A \neq B$ & $A \cap B$ & $\lnot \lnot p = p$ & $\frac{1}{3+7}$ & $f(x) = x^3$ \\
 $v_8$ & $3x - 7 = 0$ & $A \times B$ & $A \neq B$ & $A \cap B$ & $\lnot \lnot p = p$ & $\frac{1}{3+7}$ & $f(x) = x^3$ \\
 $v_9$ & $3x - 7 = 0$ & $A \times B$ & $A \neq B$ & $A \cap B$ & $\lnot \lnot p = p$ & $\frac{1}{3+7}$ & $f(x) = x^3$ \\
 $v_{10}$ & $3x - 7 = 0$ & $A \times B$ & $A \neq B$ & $A \cap B$ & $\lnot \lnot p = p$ & $\frac{1}{3+7}$ & $f(x) = x^3$ \\
 $v_{11}$ & $3x - 7 = 0$ & $A \times B$ & $A \neq B$ & $A \cap B$ & $\lnot \lnot p = p$ & $\frac{1}{3+7}$ & $f(x) = x^3$ \\
 $v_{12}$ & $3x - 7 = 0$ & $A \times B$ & $A \neq B$ & $A \cap B$ & $\lnot \lnot p = p$ & $\frac{1}{3+7}$ & $f(x) = x^3$ \\
 $v_{13}$ & $3x - 7 = 0$ & $A \times B$ & $A \neq B$ & $A \cap B$ & $\lnot \lnot p = p$ & $\frac{1}{3+7}$ & $f(x) = x^3$ \\
 \hline
\end{tabular}
\end{center}

